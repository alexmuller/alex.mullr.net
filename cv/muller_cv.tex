% Alex Muller - Curriculum Vitae
%
% LaTeX layout Copyright (C) 2004-2011 Jason R. Blevins
% (with modifications)
% http://jblevins.org/projects/cv-template/
%
% You may use use this document as a template to create your own CV
% and you may redistribute the source code freely.  No attribution is
% required in any resulting documents.  I do ask that you please leave
% this notice and the above URL in the source code if you choose to
% redistribute this file.

\documentclass[10pt,a4paper]{article}

\usepackage[utf8]{inputenc}

\usepackage{hyperref}
\usepackage{geometry}

% Fonts
\usepackage[T1]{fontenc}
\usepackage[urw-garamond]{mathdesign}

% Name
\def\name{Alex Muller}

% Link colour and metadata
\hypersetup{
  colorlinks = true,
  urlcolor = [rgb]{0.2,0.2,0.6},
  pdfauthor = {\name},
  pdfkeywords = {alexmuller, york, web, glaxosmithkline},
  pdftitle = {\name: Curriculum Vitae},
  pdfsubject = {Curriculum Vitae},
  pdfpagemode = UseNone
}

\geometry{
  left   = 0.8in,
  top    = 0.8in,
  right  = 0.8in,
  bottom = 0.8in
}

% Customize page headers
\pagestyle{myheadings}
\markright{\name}
\thispagestyle{empty}

% Custom section fonts
\usepackage{sectsty}
%\sectionfont{\rmfamily\mdseries\Large}
%\subsectionfont{\rmfamily\mdseries\itshape\large}
\sectionfont{\rmfamily\mdseries\LARGE}
\subsectionfont{\rmfamily\mdseries\large}

% Other possible font commands include:
% \ttfamily for teletype,
% \sffamily for sans serif,
% \bfseries for bold,
% \scshape for small caps,
% \normalsize, \large, \Large, \LARGE sizes.

% UK date and time
% \usepackage{UKenglish}

% Don't indent paragraphs...
\setlength\parindent{0em}

% ...but do space them a little more than normal
\setlength{\parskip}{0.3em plus 0.15em minus 0.15em}

% Make lists without bullets and compact spacing
\renewenvironment{itemize}{
  \begin{list}{}{
    \setlength{\leftmargin}{1em}
    \setlength{\itemsep}{0em}
    \setlength{\parskip}{0pt}
    \setlength{\parsep}{0.2em}
  }
}{
  \end{list}
}

\begin{document}

{\huge \name}

\bigskip

\begin{minipage}[t]{0.26\textwidth}
  \textbf{Address} \\
  9 Grosvenor Road \\
  London, W4 4EJ \\
  07525 370 388 \\
\end{minipage}
\begin{minipage}[t]{0.5\textwidth}
  \textbf{Online} \\
  \href{mailto:alex@mullr.net}{alex@mullr.net} \\
  \href{http://alex.mullr.net/}{http://alex.mullr.net/} \\
  \href{https://github.com/alexmuller}{https://github.com/alexmuller} \\
\end{minipage}

\section*{Profile and interests}

I write software for the web that fulfills a user need. I use simple, well
understood technologies that are appropriate for the task in order to create
small apps that can be loosely joined via HTTP APIs by machines and
progressively enhanced on the front end to create an excellent user experience.

From the Computer Science part of my degree I gained familiarity with
fundamental concepts such as object-oriented software development, algorithms
and data structures. From Maths I developed an interest in statistical theory
that has been with me since school.

I am also interested in the creation of media including television, film and
radio, especially in how technology is aiding distribution. I am passionate
about using the web to enable free and open access to information,
particularly through contributing to projects like Wikipedia and
OpenStreetMap. Recently I have found myself increasingly interested in
\href{http://alex.mullr.net/blog/2011/11/the-web-growing-up/}{preserving the
history of the web} and considering issues like link rot.

SC clearance until June 2023 \textbullet{}
British passport \textbullet{}
UK driving licence

% Teaching? Mentoring? YSIS etc.

\section*{Work and experience}

\subsection*{Software developer, Government Digital Service (May 2013--)}

Working on the Performance Platform, helping government service owners access
the data behind their transactions. Making use of progressive enhancement to
display graphs in as many browsers as possible. Participating in the second
line (technical) support rotation for all GDS software and services,
responding to alerts from \href{https://www.gov.uk/}{GOV.UK} production
systems and improving infrastructure.

\begin{itemize}
  \item \textbf{Tools:} Python, Ruby on Rails, MongoDB, JavaScript, Puppet, Vagrant
  \item \textbf{On the web:} \url{https://www.gov.uk/performance}
\end{itemize}

\subsection*{Research and Development software engineer, News International (July 2012--May 2013)}

After graduating, I joined the small R\&D team at News International in
London. Using a variety of tools (including Ruby and Node.js) I helped create
lightweight prototypes and proof-of-concept apps for \emph{The Sun}, \emph{The
Times} and \emph{The Sunday Times}, as well as monitoring and maintaining some
apps in production. My favourite project was a marketing campaign titled Feel
Good Piñata, a Node.js app with a real-world component that used WebSockets to
provide a queueing system for users. I have experimented with responsive
design as well as designing for different media such as Internet-connected
receipt printers.

\begin{itemize}
  \item \textbf{Tools:} Ruby on Rails, Haml, Sass, JavaScript, Heroku, EC2, Node.js
  \item \textbf{On the web:} \url{http://web.archive.org/web/201301/http://labs.newsint.co.uk/showcase}
\end{itemize}

\subsection*{Python and front-end web developer, GlaxoSmithKline R\&D (July 2011--September 2011)}

Over the summer between my industrial placement and final year at university I
maintained an internal web application at GSK's Medicine Research Centre in
Stevenage. The application was originally built predominantly using Python and
JavaScript, and I was responsible for fixing bugs and adding new functionality
as required.

\subsection*{Web \& multimedia communications placement, GlaxoSmithKline (July 2010--July 2011)}

During my placement year I worked on
\href{http://www.gsk.com/}{GlaxoSmithKline's corporate website} as a member of
their Global Media team. The placement involved writing HTML, CSS and
JavaScript (primarily using jQuery) and working with non-technical
stakeholders to understand and implement their requests. I was responsible for
gathering information in preparation for a project to redesign the site, for
example through interviewing users and creating surveys. A reference is
available from \href{mailto:GSKUK.HR@acs-inc.com}{\mbox{GSKUK.HR@acs-inc.com}}.

% mbox prevents hyphenation

\subsection*{One Click Orgs (2010--2011)}

OCO is an online service to help groups with their legal, decision making and
membership structure. I regularly worked over IRC and in person with other
volunteer developers,
\href{https://github.com/oneclickorgs/one-click-orgs/commits/master?author=alexmuller}{committing}
Ruby on Rails, Haml and Sass code to the
\href{https://github.com/oneclickorgs/one-click-orgs}{OCO GitHub repository}.

\subsection*{York Students in Schools (January 2009--February 2010)}

I was involved with the York Students in Schools placement scheme, spending
half a day each week at a local school to assist the teacher with activities
and any computing issues or questions. I feel that the placement was useful in
many respects, primarily because it greatly improved my explanatory and
presentation skills.

\subsection*{Web developer, Nouse (2008--2009) and Web designer, YSTV (2009--2011)}

At university I became very involved with media societies, including the
student newspaper Nouse and the television station YSTV. Redeveloping the
newspaper's website was invaluable experience for working in a team,
reinforcing obvious good practices such as detailed code commenting. YSTV has
offered similar experiences, though there what I found most useful were the
weekly ``station meetings'', where I was able to give opinions and be
part of a group that makes decisions about the future of the station.

\section*{Education}

\subsection*{BSc Computer Science and Mathematics (with a year in industry), \\
  University of York, 2008--2012}

\begin{itemize}
  \item \textbf{Degree classification:} Second Class Honours, Division One (2:1)
  \item \textbf{Dissertation:}
    \href{http://alex.mullr.net/blog/2012/03/constrained-optimisation-allocate-modules-york/}{
      Allocating optional modules to University of York students
    } (79\%)
\end{itemize}

My final-year project involved the creation of an application that allocates
optional modules to students after collecting their preferences via a web
interface. The creation of this application was supported by the University of
York's University Teaching Committee.

The project covered two broad areas of computer science; the construction of
the web application (which included gathering requirements, database design,
user experience, testing \& security), and performing the allocation based on
students' preferences and factors like the number of students in a class,
which is a constrained optimisation problem.

The application was trialled in March 2012 by the Archaeology \& History
departments and successfully allocated modules to 800 of their students. The
pilot was evaluated as successful and the application was used in the
following academic year, supported by the University's IT Services.

A reference for this project is available from
\href{http://www-users.cs.york.ac.uk/~jc/}{James Cussens}, Department of
Computer Science.


\medskip

\begin{small}
  Favourite modules:
  \begin{itemize}
    \item Probability Theory I (first year)
    \item Design of Interactive Systems (second year)
    \item Applied Probability (final year)
    \item Formal Languages and Automata (final year)
    \item Code Generation \& Optimisation (final year)
  \end{itemize}
\end{small}

\subsection*{St Paul's School, London, 2003--2008}

\begin{itemize}
  \item \textit{A Level:} Computing (A), Mathematics (A), Physics (C)
  \begin{small}
    \item \textit{AS Level:} French (A)
  \end{small}
\end{itemize}

\section*{Reference}

\href{http://preoccupations.org/}{David Smith} \\
Director of ICT, St Paul's School

\begin{small}
  \emph{Contact details available on request}
\end{small}

\medskip

% Footer
\begin{flushright}
  \begin{footnotesize}
    Latest version available at \url{http://alex.mullr.net/cv/} \\
    This copy created \today
  \end{footnotesize}
\end{flushright}

\end{document}
