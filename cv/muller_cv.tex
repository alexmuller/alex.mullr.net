% Alex Muller - Curriculum Vitae
%
% LaTeX layout Copyright (C) 2004-2011 Jason R. Blevins
% (with modifications)
% http://jblevins.org/projects/cv-template/
%
% You may use use this document as a template to create your own CV
% and you may redistribute the source code freely.  No attribution is
% required in any resulting documents.  I do ask that you please leave
% this notice and the above URL in the source code if you choose to
% redistribute this file.

\documentclass[10pt,a4paper]{article}

\usepackage[utf8]{inputenc}

\usepackage{hyperref}
\usepackage{geometry}

% Fonts
\usepackage[T1]{fontenc}
\usepackage[urw-garamond]{mathdesign}

% Name
\def\name{Alex Muller}

% Link colour and metadata
\hypersetup{
  colorlinks = true,
  urlcolor = [rgb]{0.1,0.2,0.4},
  pdfauthor = {\name},
  pdfkeywords = {alexmuller, york, web, glaxosmithkline},
  pdftitle = {\name: Curriculum Vitae},
  pdfsubject = {Curriculum Vitae},
  pdfpagemode = UseNone
}

\geometry{
  left   = 0.8in,
  top    = 0.8in,
  right  = 0.8in,
  bottom = 0.8in
}

% Customize page headers
\pagestyle{myheadings}
\markright{\name}
\thispagestyle{empty}

% Custom section fonts
\usepackage{sectsty}
%\sectionfont{\rmfamily\mdseries\Large}
%\subsectionfont{\rmfamily\mdseries\itshape\large}
\sectionfont{\rmfamily\mdseries\LARGE}
\subsectionfont{\rmfamily\mdseries\large}

% Other possible font commands include:
% \ttfamily for teletype,
% \sffamily for sans serif,
% \bfseries for bold,
% \scshape for small caps,
% \normalsize, \large, \Large, \LARGE sizes.

% UK date and time
% \usepackage{UKenglish}

% Don't indent paragraphs...
\setlength\parindent{0em}

% ...but do space them a little more than normal
\setlength{\parskip}{0.3em plus 0.15em minus 0.15em}

% Make lists without bullets and compact spacing
\renewenvironment{itemize}{
  \begin{list}{}{
    \setlength{\leftmargin}{1.5em}
    \setlength{\itemsep}{0.25em}
    \setlength{\parskip}{0pt}
    \setlength{\parsep}{0.25em}
  }
}{
  \end{list}
}

\begin{document}

{\huge \name}

\bigskip

\begin{minipage}[t]{0.26\textwidth}
  \textbf{Address} \\
  9 Grosvenor Road \\
  London, W4 4EJ \\
  07525 370 388 \\
\end{minipage}
\begin{minipage}[t]{0.5\textwidth}
  \textbf{Online} \\
  \href{mailto:alex@mullr.net}{alex@mullr.net} \\
  \href{http://alex.mullr.net/}{http://alex.mullr.net/} \\
  \href{https://github.com/alexmuller}{https://github.com/alexmuller} \\
\end{minipage}

\section*{Profile and interests}

I am very interested in the technical side of both the computing and
communications industries and enjoy working on the user-facing design of
software and systems (human-computer interaction). From my Computer Science
course, I gained familiarity with fundamental concepts such as object-oriented
programming, algorithms and data structures, and had experience writing
Java and Python. From Mathematics I have developed an interest in statistical
theory that has been with me since school, choosing half of my Maths modules
to be related to probability and statistics.

I am also interested in the creation of media including television, film and
radio, as well as how technology is helping to distribute these kinds of
content in new ways. I am passionate about using the web to enable free and
open access to information, particularly through contributing to projects like
Wikipedia and OpenStreetMap. Recently I have found myself increasingly
interested in
\href{http://alex.mullr.net/blog/2011/11/the-web-growing-up/}{preserving the
history of the web} and considering issues like link rot.

% Teaching? Mentoring? YSIS etc.

\section*{Previous work and experience}

% Further examples of my work and experience are available at
% \href{http://alex.mullr.net/cv/}{http://alex.mullr.net/cv/}.

\subsection*{Research and Development software engineer, News International (July 2012--)}

\emph{Tools: Ruby on Rails, Haml, Sass, JavaScript, Heroku, Node.js}

After graduating, I joined the small R\&D team at News International in
London. Using a variety of tools (including Ruby and Node.js) I help create
lightweight prototypes and proof-of-concept apps for \emph{The Sun}, \emph{The
Times} and \emph{The Sunday Times}, as well as monitoring and maintaining some
apps in production. My favourite project so far is a marketing campaign titled Feel
Good Piñata, a Node.js app using WebSockets to provide a queueing system for
users with authentication via Twitter and Facebook. I have experimented with
responsive design as well as designing for different media such as receipt
printers. More information is available at
\href{http://labs.newsint.co.uk/}{labs.newsint.co.uk}.

\subsection*{Python and front-end web developer, GlaxoSmithKline R\&D (July 2011--September 2011)}

Over the summer between my industrial placement and final year at university I
maintained an internal web application at GSK's Medicine Research Centre in
Stevenage. The application was originally built predominantly using Python and
JavaScript, and I was responsible for fixing bugs and adding new functionality
as required.

\subsection*{Web \& multimedia communications placement, GlaxoSmithKline (July 2010--July 2011)}

During my placement year I worked on
\href{http://www.gsk.com/}{GlaxoSmithKline's corporate website} as a member of
their Global Media team. The placement involved writing HTML, CSS and
JavaScript (primarily using jQuery) and working with non-technical
stakeholders to understand and implement their requests. I was responsible for
gathering information in preparation for a project to redesign the site, for
example through interviewing users and creating surveys. A reference is
available from \href{mailto:GSKUK.HR@acs-inc.com}{\mbox{GSKUK.HR@acs-inc.com}}.

% mbox prevents hyphenation

\subsection*{One Click Orgs (April 2010--)}

OCO is an online service to help groups with their legal, decision making and
membership structure. I regularly work over IRC and in person with other
volunteer developers, committing Ruby on Rails, Haml and Sass code to the
\href{https://github.com/oneclickorgs/one-click-orgs/}{OCO GitHub repository}.

\subsection*{York Students in Schools (January 2009--February 2010)}

I was involved with the York Students in Schools placement scheme, spending
half a day each week at a local school to assist the teacher with activities
and any computing issues or questions. I feel that the placement was useful in
many respects, primarily because it greatly improved my explanatory and
presentation skills.

\subsection*{Web developer, Nouse (2008--2009) and Web designer, YSTV (2009--2011)}

At university I became very involved with media societies, including the
student newspaper Nouse and the television station YSTV. Redeveloping the
newspaper's website was invaluable experience for working in a team,
reinforcing obvious good practices such as detailed code commenting. YSTV has
offered similar experiences, though there what I found most useful were the
weekly ``station meetings'', where I was able to give opinions and be
part of a group that makes decisions about the future of the station.

\section*{Education}

\subsection*{BSc Computer Science and Mathematics (with a year in industry), \\
  University of York, 2008--2012}

\begin{itemize}
  \item \textit{Degree classification:} Second Class Honours, Division One (2:1)
  \item \textit{Dissertation:}
    \href{http://alex.mullr.net/blog/2012/03/constrained-optimisation-allocate-modules-york/}{
      Allocating optional modules to University of York students
    } (79\%)
\end{itemize}

My final-year project involved the creation of an application that allocates
optional modules to students after collecting their preferences via a web
interface. The creation of this application was supported by the University of
York's University Teaching Committee.

The project covered two broad areas of computer science; the construction of
the web application (which included gathering requirements, database design,
user experience, testing \& security), and performing the allocation based on
students' preferences and factors like the number of students in a class,
which is a constrained optimisation problem.

The application was trialled in March 2012 by the Archaeology \& History
departments and successfully allocated modules to 800 of their students. It is
hoped that based on the success of this pilot, the University's IT Services
will be able to offer the application to all departments in the 2013 academic
year.

A reference for this project is available from
\href{http://www-users.cs.york.ac.uk/~jc/}{James Cussens}, Department of
Computer Science.


\medskip

% \begin{small}
%   Favourite modules:
%   \begin{itemize}
%     \item Probability Theory I (first year)
%     \item Introduction to Group Theory (second year)
%     \item Design of Interactive Systems (second year)
%     \item Applied Probability (final year)
%     \item Formal Languages and Automata (final year)
%     \item Code Generation \& Optimisation (final year)
%   \end{itemize}
% \end{small}

\begin{small}
\begin{minipage}[t]{0.5\textwidth}

\textit{Modules studied in the second year:}

\begin{tabular}{ p{6cm} p{1cm} }
  Theory of Computing                          \\ % & 48\% \\
  Design of Interactive Systems                \\ % & 50\% \\
  Logic Programming \& Artificial Intelligence \\ % & 51\% \\
  Computer Graphics and Visualisation          \\ % & 43\% \\
  Vector Calculus I                            \\ % & 54\% \\
  Statistical Theory I                         \\ % & 14\% \\
  Statistical Theory II                        \\ % & 55\% \\
  Statistical Theory III                       \\ % & 62\% \\
  Analysis I                                   \\ % & 47\% \\
  Introduction to Group Theory                 \\ % & 67\% \\
\end{tabular}

\end{minipage}
\begin{minipage}[t]{0.5\textwidth}

\textit{Modules studied in the final year:}

\begin{tabular}{ p{6cm} p{1cm} }
  Final-year project                 \\ % & 79\% \\
  Algorithms for Graphical Models    \\ % & 74\% \\
  Crypto, Attacks \& Countermeasures \\ % & 65\% \\
  Code Generation \& Optimisation    \\ % & 83\% \\
  Applied Probability                \\ % & 86\% \\
  Introduction to Number Theory      \\ % & 72\% \\
  Generalised Linear Models          \\ % & 59\% \\
  Formal Languages and Automata      \\ % & 84\% \\
  Bayesian Statistics                \\ % & 27\% \\
\end{tabular}

\end{minipage}
\end{small}

\subsection*{St Paul's School, London, 2003--2008}

\begin{itemize}
  \item \textit{A Level:} Computing (A), Mathematics (A), Physics (C)
  \begin{small}
    \item \textit{AS Level:} French (A)
  \end{small}
\end{itemize}

\section*{Reference}

\href{http://preoccupations.org/}{David Smith} \\
Director of ICT, St Paul's School \\
\emph{Contact details available on request} \\

\medskip

% Footer
\begin{flushright}
  \begin{footnotesize}
    Latest version available at \url{http://alex.mullr.net/cv/} \\
    This copy created \today
  \end{footnotesize}
\end{flushright}

\end{document}
